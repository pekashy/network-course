\documentclass[a4paper,12pt]{article}
\usepackage{amsmath,amssymb,amsfonts,amsthm}
\usepackage{tikz}
\usepackage [utf8x] {inputenc}
\usepackage [T2A] {fontenc} 
\usepackage[russian]{babel}
\usepackage{cmap} 
\usepackage{ gensymb }
% Так ссылки в PDF будут активны
\usepackage[unicode]{hyperref}
\usepackage{ textcomp }
\usepackage{indentfirst}
\usepackage[version=3]{mhchem}

% вы сможете вставлять картинки командой \includegraphics[width=0.7\textwidth]{ИМЯ ФАЙЛА}
% получается подключать, как минимум, файлы .pdf, .jpg, .png.
\usepackage{graphicx}
% Если вы хотите явно указать поля:
\usepackage[margin=1in]{geometry}
% Или если вы хотите задать поля менее явно (чем больше DIV, тем больше места под текст):
% \usepackage[DIV=10]{typearea}

\usepackage{fancyhdr}

\newcommand{\bbR}{\mathbb R}%теперь вместо длинной команды \mathbb R (множество вещественных чисел) можно писать короткую запись \bbR. Вместо \bbR вы можете вписать любую строчку букв, которая начинается с '\'.
\newcommand{\eps}{\varepsilon}
\newcommand{\bbN}{\mathbb N}
\newcommand{\dif}{\mathrm{d}}

\newtheorem{Def}{Определение}
\title{Климанов. Семинар 1}
\begin{document}
	\maketitle
\subsection{Некоторое оборудование, которое есть в классе}
\begin{enumerate}
	\item 2960 - классические l2-switch (cysco catalyst)
	\item 3750 (светленькие в стойке)
	\item 3550 - тоже в стойке
	\item 2511 - цисковские консольные манипуляторы
	
\end{enumerate}

\subsection{Схема управления OOB (OUT OF BAND)}
Никакие пользователшьские данные в этой схеме не передаются, только настройки

На задней панели роутера есть порт, который называется console, подразумевалось, что через этот порт роутер подключается к com-порту компьютераю
Раньше никакой адресации не было, устройство напрямую подключалось к сщь-порту. Выставлялась битность, скорость и т.д.
Теперь начали использоваться usb-порты.

Со стороны сетевого оборудования используется на ethernet (8p8c - 8 пос. мест, 8 контактов), часто называется RJ-45(registered jack), но это не очень верно, тк распиновка не соответствует RJ45.
На совсем современном оборудовании используется mini-usb

binband - используется обычно на пользовательских моделях, один кабель для данных и трафика. Считается менее безопасным, менее надежным. Обычно все реализовано через него, но существует экстренная сеть управления.

mgmt - игнтерфейс, который используется только для управления.

На конкретном оборудовании может быть все 3 способа управления, на домашнем обычно 1.

\subsection{title}

На оборудовании Ethernet-вида интерфейсы - под подключение rj-45

\subsubsection{Оптические интерфейсы}
\begin{enumerate}
	\item ?
	\item sfp
	\item usfp
	\item e1 - похожие на Rj-45, но другие, не совместимы
	\item serial-интерфейсы
\end{enumerate}

\subsubsection{Как маркируютсяы интерфейсы:}
\begin{enumerate}
	\item Сначала пишется технология (e.g. Fe, Fag1, ten, serial)
	\item Цифры (устройства бывают с изменяемой и с фиксированной конфигурацией, может быть возможно поставить модули, содержащие интерфейсы (модульные устройства)). Нумерация:
	\subitem номер слота
	\subitem номер интерфейса в слоте
	\subitem номер модуля\\
	e.g. Fa101
	
	В устройствах с фиксированной конфигурацией (коммутаторы 3750 - \emph{стекируемая} модель, можно соединять в стек при помощи стек-портов, что позволяет ими управлять как единым целым, в стек можно объединить более 2ух коммутаторов, stack-ring - кольцо соединенных коммутаторов), тогда первая цифра значит не номер слота, а номер коммутатора в стеке.
	
\end{enumerate}

\section{HW}
Олифер на предмет OSI

ACMD На предмет коммутаторов

\end{document}
