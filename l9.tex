\documentclass[a4paper,10pt]{article}
\usepackage[table,xcdraw]{xcolor}
\usepackage{multirow}
\usepackage{amsmath,amssymb,amsfonts,amsthm}
\usepackage{tikz}
\usepackage [utf8x] {inputenc}
\usepackage [T2A] {fontenc} 
\usepackage[russian]{babel}
\usepackage{cmap} 
\usepackage{ gensymb }
% Так ссылки в PDF будут активны
\usepackage[unicode]{hyperref}
\usepackage{ textcomp }
\usepackage{indentfirst}
\usepackage{adjustbox}
\usepackage[version=3]{mhchem}
\usepackage{MnSymbol,wasysym}

% вы сможете вставлять картинки командой \includegraphics[width=0.7\textwidth]{ИМЯ ФАЙЛА}
% получается подключать, как минимум, файлы .pdf, .jpg, .png.
\usepackage{graphicx}
% Если вы хотите явно указать поля:
\usepackage[margin=1in]{geometry}
% Или если вы хотите задать поля менее явно (чем больше DIV, тем больше места под текст):
\usepackage[DIV=20]{typearea}

\usepackage{fancyhdr}

\newcommand{\bbR}{\mathbb R}%теперь вместо длинной команды \mathbb R (множество вещественных чисел) можно писать короткую запись \bbR. Вместо \bbR вы можете вписать любую строчку букв, которая начинается с '\'.
\newcommand{\eps}{\varepsilon}
\newcommand{\bbN}{\mathbb N}
\newcommand{\dif}{\mathrm{d}}

\newtheorem{Def}{Определение}
\title{Климанов. Лекция 9}

\begin{document}
	\maketitle
	
	...(современный TCP)...
	
	\section{Оптимизации TCP}
	\paragraph{Алгоритм Нейгла} Если на высоких уровнях есть протокол, который оперирует небольшими порциями данных (нельзя сформировать сегмент сразу), TCP может собирать их, а потом отправить под одних заголовком
	
	\paragraph{Delayed ACK} Подтверждение отправляется внагрузку к данным, которые нужно отправить, это позволяет не отправлять пустые сегменты и снизить нагрузку на сеть
	
	\section{UDP}
	
	Заголовок:
	
	\begin{table}[h]
		\begin{tabular}{|l|l|}
			\hline
			Source Port & Dest Port \\ \hline
			Length      & Check Sum \\ \hline
		\end{tabular}
	\end{table}

	\textit{При открытии порта делается запрос в систему, указывается тип протокола, система открывает порт и передает данные приложению}
	
	\paragraph{TFTP} - протокол, сделанный на основе UDP, в котором реализованы средства, гарантирующие доставку (но ограничений сильно больше, чем в TCP).
	
	\section{ICMP}
	Протокол управляющихх сообщений для протокола IP.
	
	Заголовок [8 bytes]:
	
	\begin{table}[h]
		\begin{tabular}{|l|l|l|l|}
			\hline
			Type {[}1 byte{]} & Code {[}1 byte{]} & \multicolumn{2}{l|}{Checksum {[}2 bytes{]}} \\ \hline
		\end{tabular}
	\end{table}

	Используется для доставки сообщений об ошибках (нет маршрута, пакет слишком большой, TTL закончился (\texttt{TTL exceeded})).
	
	\section{VLAN}
	VLAN - Virtual Local Area Network.
	Порты в \emph{управляемых коммутаторах} можно объединять в группы, для каждой будет отдельная изолированная таблица коммутации.
	
	Принадлежность к VLAN'у определяется на основе принадлежности к физическому интерфейсу, либо на основе тегов.
	
	\paragraph{IEEE 802.1Q/P} - стандарт, документирующий тегирование для коммутаторов (Q - набор стандартов для маркировки, P набор стандартов для - качества обслуживания)
	\paragraph{ISL} - цисковсский стандарт, документирующий тегирование (уст. тк был менее эффективен)
	\paragraph{Trunk} - канал или интерфейс, через которые передаются данные нескольких VLAN-сетей.
	
	\subsection{IEEE 802.1Q/P}
	Заголовок:
\begin{table}[h]
	\begin{tabular}{|l|l|l|l|}
		\hline
		\begin{tabular}[c]{@{}l@{}}IPID {[}16 bit{]} - какой протокол\\  используется для тегирования\end{tabular} & PCP {[}3 bit{]} - приоритет трафика & \begin{tabular}[c]{@{}l@{}}CFI {[}1 bit{]} - идентификатор\\  каноничности формата (Ethernet или нет\end{tabular} & \begin{tabular}[c]{@{}l@{}}VID {[}12 bit{]} - какому\\  VLAN принадлежит\end{tabular} \\ \hline
	\end{tabular}
\end{table}
\end{document}