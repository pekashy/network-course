\documentclass[a4paper,10pt]{article}
\usepackage[table,xcdraw]{xcolor}
\usepackage{multirow}
\usepackage{amsmath,amssymb,amsfonts,amsthm}
\usepackage{tikz}
\usepackage [utf8x] {inputenc}
\usepackage [T2A] {fontenc} 
\usepackage[russian]{babel}
\usepackage{cmap} 
\usepackage{ gensymb }
% Так ссылки в PDF будут активны
\usepackage[unicode]{hyperref}
\usepackage{ textcomp }
\usepackage{indentfirst}
\usepackage{adjustbox}
\usepackage[version=3]{mhchem}

% вы сможете вставлять картинки командой \includegraphics[width=0.7\textwidth]{ИМЯ ФАЙЛА}
% получается подключать, как минимум, файлы .pdf, .jpg, .png.
\usepackage{graphicx}
% Если вы хотите явно указать поля:
\usepackage[margin=1in]{geometry}
% Или если вы хотите задать поля менее явно (чем больше DIV, тем больше места под текст):
\usepackage[DIV=20]{typearea}

\usepackage{fancyhdr}

\newcommand{\bbR}{\mathbb R}%теперь вместо длинной команды \mathbb R (множество вещественных чисел) можно писать короткую запись \bbR. Вместо \bbR вы можете вписать любую строчку букв, которая начинается с '\'.
\newcommand{\eps}{\varepsilon}
\newcommand{\bbN}{\mathbb N}
\newcommand{\dif}{\mathrm{d}}

\newtheorem{Def}{Определение}
\title{Климанов. Лекция 7}

\begin{document}
	\maketitle
	\section{Маршрутизация}
	Таблица маршрутизации имеет вид:
	
	\begin{table}[h]
		\begin{tabular}{llll}
			Source & Subnet      & mask        & next\_hop \\
			S      & 192.168.0.0 & 255.255.0.0 & 10.1.1.1  \\
			C      &             &             & Fa0/0    
		\end{tabular}
	\end{table}

	Записи, который появляются в таблице автоматически помечаются отдельно (в оборудовании Cisco буквой C), а в \emph{next-hop} появляется название интерфейса.
	
	Также есть записи, заносимые в таблицу вручную [S]
	
	Также есть динамические записи (буква в зависимости от протокола)
	
	\subsection{Маршрутизирующие протоколы}
	
	Благодаря маршрутизирующим протоколам появляются записи в таблице маршрутизации
	
	\subsubsection{Внутренние}
	\texttt{IGP, RIP, OSPF, EIGRP, IS-IS}
	\subsubsection{Внешние}
	\texttt{EGP, BGP}
	\\
	
	\textbf{В} случае наличия двух записей в таблице маршрутизации для одного адреса берется запись с \underline{более длинной маской} (для меньшего числа узлов, более конкретная)
	
	В случае наличия двух записей от разных источников с одинаковым адресом, применяется переменная \texttt{AD} - Administrative Distance (\textit{степень доверия источнику информации, меньше - больше доверия})
	
	\subsubsection{Administrative Distance}
	
	Static - 1
	
	Connected - 0
	
	RIP -120
	
	OSPF - 110
	
	Свои значения Administrative distance могут быть заданы системным администратором
	
	Таблица маршрутизации составляется в результате работы маршрутизирующих протоколов до прихода пользовательских данных.
	
	\subsection{Широковещание}
	
	Существует два вида широковещания: \emph{общее} и \emph{направленное}.
	
	\textbf{Общее} широковещание роутер не маршрутизирует.
	
	\textbf{Направленное} широковещание в принципе является широковещанием только в сети назначения (пакет \texttt{10.1.1.255} идет до роутера \texttt{10.1.1.1/24} как обычный, а уже там может стать широковещательным)
	\newpage
	\section{Протокол ARP}
	\textbf{ARP} - \emph{adress resolution protocol} позволяет по IP адресу узнать MAC-адрес.
	
	% Please add the following required packages to your document preamble:
	% \usepackage{multirow}
	\begin{table}[h]
		\begin{tabular}{ccl}
			byte-\textgreater{} & 0                                   & \multicolumn{1}{c}{1}                 \\
			1                   & \multicolumn{2}{c}{{[}HType{]} Hardware type (какая среда используется)}    \\
			2                   & \multicolumn{2}{c}{{[}PTYPE{]} Protocol type (какой протокол используется)} \\
			3                   & HLEN -длина HTYPE                   & PLEN - длина PTYPE                    \\
			4                   & \multicolumn{2}{c}{OPER - operation}                                        \\
			5                   & \multicolumn{2}{c}{\multirow{3}{*}{Sha Sender hardware address}}            \\
			6                   & \multicolumn{2}{c}{}                                                        \\
			7                   & \multicolumn{2}{c}{}                                                        \\
			8                   & \multicolumn{2}{c}{\multirow{2}{*}{SPA Sender protocol address}}            \\
			9                   & \multicolumn{2}{c}{}                                                        \\
			10,11,12            & \multicolumn{2}{c}{Tha {[}x3{]} Target hardware address}                    \\
			13,14,15            & \multicolumn{2}{c}{TPA {[}x2{]} Target protocol address}                   
		\end{tabular}
	\caption{Вид ARP-запроса}
	\end{table}

	Весть траффик, который адрес не знает, куда отправить, он отправляет на Default-gateway. Если узел хоть что-то "знает" (он может проанализировать IP-адрес и маску), то он знает адрес сети, к которой подключен
	
	\section{Процесс установления соединения}
	\begin{figure}[h]
		\includegraphics[width=0.7\linewidth]{7-1}
	\end{figure}
	
	1. Берутся адреса и маски своя и получателя, они сопоставляются, сверяются адреса сети.
	
	Если адреса сети совпадают:
	
	2. Формируется ARP-запрос, отправляется всем узлам сети ARP-запрос
	
	3. Все узлы кроме нужного игнорируют, нужный отправляет ответный ARP-запрос
	
	4. После этого узлы готовы к обмену данными
	
	Если адреса подсетей \underline{не совпадают}:
	2. Отправитель обращается к узлу по умолчанию (который на картинке должен указывать на Router0 (или провайдеру))
	
	3. Отправляется ARP-запрос на адрес роутера, в ответ получаем MAC-адрес интерфейса роутера Router0, к которому подключена подсеть
	
	4. Отправляется пакет \underline{с адресом отправителя - MAC PC0, MAC адрес получателя - адрес роутера получателя Router1}.
	
	5. Router1 видит, что IP - адрес отправителя не совпадает с его адресом, он пересылает его дальше на hext\_hop.
	
	6. Формирует ARP-запрос, узнает MAC-адрес \texttt{10.0.2.0/24}. Отправляет.
	
	7. \texttt{10.0.2.0/24} понимает, что пакет все-таки тоже не ему, он аналогично ARP-запросом узнает адрес PC3
	
	\textit{В случае если на узле IP-адрес задан статически, при включении проверяется занятость IP-адреса (формируется ARP-запрос с адресом назначения в виде его собственного IP-адреса). Если он получает ответ, выдается ошибка duplicate IP adress и адрес не используется.}
	
	\textit{При выполнении этого самого изначального ARP-запроса заодно обучаются все коммутаторы в сети}.
\end{document}