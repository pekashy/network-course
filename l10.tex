\part{\documentclass[a4paper,10pt]{article}
\usepackage[table,xcdraw]{xcolor}
\usepackage{multirow}
\usepackage{amsmath,amssymb,amsfonts,amsthm}
\usepackage{tikz}
\usepackage [utf8x] {inputenc}
\usepackage [T2A] {fontenc} 
\usepackage[russian]{babel}
\usepackage{cmap} 
\usepackage{ gensymb }
% Так ссылки в PDF будут активны
\usepackage[unicode]{hyperref}
\usepackage{ textcomp }
\usepackage{indentfirst}
\usepackage{adjustbox}
\usepackage[version=3]{mhchem}
\usepackage{MnSymbol,wasysym}

% вы сможете вставлять картинки командой \includegraphics[width=0.7\textwidth]{ИМЯ ФАЙЛА}
% получается подключать, как минимум, файлы .pdf, .jpg, .png.
\usepackage{graphicx}
% Если вы хотите явно указать поля:
\usepackage[margin=1in]{geometry}
% Или если вы хотите задать поля менее явно (чем больше DIV, тем больше места под текст):
\usepackage[DIV=20]{typearea}

\usepackage{fancyhdr}

\newcommand{\bbR}{\mathbb R}%теперь вместо длинной команды \mathbb R (множество вещественных чисел) можно писать короткую запись \bbR. Вместо \bbR вы можете вписать любую строчку букв, которая начинается с '\'.
\newcommand{\eps}{\varepsilon}
\newcommand{\bbN}{\mathbb N}
\newcommand{\dif}{\mathrm{d}}

\newtheorem{Def}{Определение}
\title{Климанов. Лекция 10}

\begin{document}
	\maketitle}
	\section{Динамическая маршрутизация}
	\paragraph{Маршрутизирутизирующие протоколы} - участвующие в маршрутизации.
	\paragraph{Маршрутизируемые протоколы} - очев.
	
	\subsection{Протокол RIP}
	Работает через протокол \emph{UDP} через 520 порт, не устанавливает соседства (сессии)
	\paragraph{Routing Information Protocol}
	Существует 2 версии:
	\paragraph{v1} - выполняет классовую маршрутизацию, без использования масок, более древний. Не используется в реальных сетях.
	\paragraph{v2} - использует маски, более новый, но все еще наследует много свойств от \textbf{v1}. Все еще используется но не очень популярен в реальных сетях.
	Также существует \textbf{RIPng} - для \emph{ipv6}
	
	\subsubsection{Метрики}
	Функция, считающая тем или иным способом расстояние.
	
	Меньшее значение метрики означает лучшую достижимость узла.
	
	В RIPv2 используется в качестве метрики число \emph{L3-переходов (между роутерами)}
	
	Максимальное значение метрики в \emph{RIPv2} - \textbf{15}.
	
	\textbf{16} - означает недостижимый узел.
	
	\textit{К сожалению, такая метрика не учитывает важные параметры сети вроде пропускной способности, задержки, загруженности и тд.}
	
	\subsection{Таймеры}
	\paragraph{UPDATE-timer} 30 секунд, частота рассылки \emph{RIP-database} всем узлам.
	\paragraph{INVALID-timer} 180 секунд - время жизни маршрута
	
	\subsection{Оптимизации протокола}
	В классическом RIP, тк UPDATE- таймеры продолжают отправляться пока не закончится invalid-таймер, в результате мы теряем по 3 минуты на каждом узле при распространении информации о недоступности сети. Что бы этого избежать, используется:
	\paragraph{Root-posing} - при получении информации о недоступности сети, роутер отправляет в датабазу метрику для этого пути \textbf{16}, база соответственно рассылается всем при UPDATE.
	
	Из-за вероятности потери пакета, был введен еще один механизм -
	\paragraph{Possibly down} - маршрут помечается как возможно недоступный уже при наступлении \textbf{Hold down}-таймера.
	
	\paragraph{Flush timer} (Cisco) - сколько времени должно пройти до удаления маршрута (default -240 сек.)
	У CISCO
	
	\paragraph{Метод split horizon}, в UPDATE, отправляющийся в определенный порт не вкладывается информация о сети, от которой пришла информация, через этот порт. (надеюсь, понятно).
	
	(тут я немного заснул)...
	
	\subsection{Native VLAN}
	В \emph{802.1P/Q} существует единственная сеть, траффик которой не теггируется
	
\end{document}