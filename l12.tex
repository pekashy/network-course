\documentclass[a4paper,10pt]{article}
\usepackage[table,xcdraw]{xcolor}
\usepackage{multirow}
\usepackage{amsmath,amssymb,amsfonts,amsthm}
\usepackage{tikz}
\usepackage [utf8x] {inputenc}
\usepackage [T2A] {fontenc} 
\usepackage[russian]{babel}
\usepackage{cmap} 
\usepackage{ gensymb }
% Так ссылки в PDF будут активны
\usepackage[unicode]{hyperref}
\usepackage{ textcomp }
\usepackage{indentfirst}
\usepackage{adjustbox}
\usepackage[version=3]{mhchem}
\usepackage{MnSymbol,wasysym}

% вы сможете вставлять картинки командой \includegraphics[width=0.7\textwidth]{ИМЯ ФАЙЛА}
% получается подключать, как минимум, файлы .pdf, .jpg, .png.
\usepackage{graphicx}
% Если вы хотите явно указать поля:
\usepackage[margin=1in]{geometry}
% Или если вы хотите задать поля менее явно (чем больше DIV, тем больше места под текст):
\usepackage[DIV=20]{typearea}

\usepackage{fancyhdr}

\newcommand{\bbR}{\mathbb R}%теперь вместо длинной команды \mathbb R (множество вещественных чисел) можно писать короткую запись \bbR. Вместо \bbR вы можете вписать любую строчку букв, которая начинается с '\'.
\newcommand{\eps}{\varepsilon}
\newcommand{\bbN}{\mathbb N}
\newcommand{\dif}{\mathrm{d}}

\newtheorem{Def}{Определение}
\title{Климанов. Лекция 12}

\begin{document}
	\maketitle
	\section{Spanning tree}
	(...)
	\section{Топологии} 
	Access Layer (Коммутаторы)\\
	|\\
	V\\
	Aggregation Layer (L3-коммутаторы, маршрутизаторы со свичами - зона, где L2 встречается с L3)\\
	|\\
	V\\
	Core (задача - максимально быстрая переддача данных)
	
	(...)
	\section{NAT/PAT}
	Технологии трансляции, которые позволяют получить доступ к глобальной сети 
	\subsection{NAT}
	\paragraph{NAT} - \emph{Network Adress Translation}
	
	\begin{figure}[h]
		\includegraphics[width=0.7\linewidth]{12-1}
	\end{figure}
	\textbf{NAT} позволяет транслировать адреса локальной сети в адреса глобальной, доступные провайдеру
	
	Допустим, мы хотим передать пакет от узла \textbf{A} к некоему серверу \textbf{B}, имеющему IP \texttt{8.8.8.8}.
	\begin{enumerate}
		\item IP=\texttt{192.168.0.2} - \textbf{inside local}
		
	\textbf{IP outside local }= \texttt{8.8.8.8} -\textbf{outside local}
		\item IP =\texttt{1.1.1.2} - \textbf{inside global}
		
	 \textbf{IP outside local} = \texttt{8.8.8.8} -\textbf{outside global}
	\end{enumerate}
	Таблица трансляции обновляется динамически
	
	\subsection{PAC}
	\paragraph{PAT} - \emph{Network Port Translation}
	В \textbf{PAT}-трансляции задействован еще и транспортный уровень TCP:
	
	Таблица трансляции PAT: 
	
	$ IP_{il}+PORT_{il} <->IP_{ig}+Port_{ig} $
\end{document}