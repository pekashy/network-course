\documentclass[a4paper,10pt]{article}
\usepackage[table,xcdraw]{xcolor}
\usepackage{multirow}
\usepackage{amsmath,amssymb,amsfonts,amsthm}
\usepackage{tikz}
\usepackage [utf8x] {inputenc}
\usepackage [T2A] {fontenc} 
\usepackage[russian]{babel}
\usepackage{cmap} 
\usepackage{ gensymb }
% Так ссылки в PDF будут активны
\usepackage[unicode]{hyperref}
\usepackage{ textcomp }
\usepackage{indentfirst}
\usepackage{adjustbox}
\usepackage[version=3]{mhchem}
\usepackage{MnSymbol,wasysym}

% вы сможете вставлять картинки командой \includegraphics[width=0.7\textwidth]{ИМЯ ФАЙЛА}
% получается подключать, как минимум, файлы .pdf, .jpg, .png.
\usepackage{graphicx}
% Если вы хотите явно указать поля:
\usepackage[margin=1in]{geometry}
% Или если вы хотите задать поля менее явно (чем больше DIV, тем больше места под текст):
\usepackage[DIV=20]{typearea}

\usepackage{fancyhdr}

\newcommand{\bbR}{\mathbb R}%теперь вместо длинной команды \mathbb R (множество вещественных чисел) можно писать короткую запись \bbR. Вместо \bbR вы можете вписать любую строчку букв, которая начинается с '\'.
\newcommand{\eps}{\varepsilon}
\newcommand{\bbN}{\mathbb N}
\newcommand{\dif}{\mathrm{d}}

\newtheorem{Def}{Определение}
\title{Климанов. Лекция 11}

\begin{document}
	\maketitle
	
	\begin{figure}[h!]
		\centering
		\includegraphics[width=0.5\linewidth]{11-1}
		\caption{Пример топологии}
	\end{figure}
	
	В сети такого вида могут начать циркулировать 2 одинаковых фрейма. В сети более сложной топологии фреймы могут начать размножаться.
	
	Из этого 2 проблемы:
	
	\begin{enumerate}
		\item Создается ненужная нагрузка на сеть, неправильно утилизируется канал
		\item Нагружается\textbf{data plane - передающий уровень}, оказывается влияние на \textbf{control plane и management plane - управляющий уровень}, нагружается CPU, постоянно происходит перестроение таблицы маршрутизации.
	\end{enumerate}

	\textit{В целом в сетях дублирование - хорошо, но нужно найти способ избежать таких колец}
	\section{STP - spanning tree protocol} 
	\paragraph{IEEE 802.1D} - Классическая реализация
	\subsection{Принцип работы}
	
	Протокол строит граф, потом разрывает петли
	
	1. Выбирается корень графа, корневой коммутатор, у которого \emph{bridge id (BID)} минимален
	\paragraph{BID} - 8байтовое число, состоящее из 2 частей: \texttt{Priority 2 байта} | \texttt{MAC-адрес коммутатора (у управляющих коммутаторов они есть)}
	 
	2. Каждый коммутатор считает себя корнем и рассылает через все порты специальные сообщения \textbf{BPDU}. Коммутатор слушает порты, после получения сообщения от коммутатора с меньшим \emph{BID}, коммутатор перестает считать себя корнем и начинает ретранслировать сообщения другого коммутатора. Таким образом схема сходится.
	
	3. Каждый коммутатор, который не является корне выбирает на основе \textbf{Root Path Cost} наикратчайший маршрут до корня. На каждом интерфейсе после получения сообщения от корня задается стоимость этого интерфейса, также в подсчете стоимости участвует пропускная способность. После получения BPDU, коммутатор добавляет стоимость интерфейса к общей.
	
	4. В случае существования колец в сети, коммутатор получит 2 или больше \textbf{Root Path Cost}, что позволит ему выбрать путь с наименьшей стоимостью - \textbf{Root Port или корневой интерфейс. На корневом коммутаторе нет \emph{Root-порта}}
	
	5. Раздаем следующие роли \textbf{designated или назначенные порты}. Designated - порты, которые передают траффик в сегмент. Получаем \emph{BPDU} от всех коммутаторов, подключенных к этому сегменту, выбираем тот коммутатор, который анонсирует \emph{root past cost}. Если \emph{Root path cost} одинаковы, выбираем с наименьшим \emph{BID} (тоже заложены в \emph{BPDU})
	
	6. Все не \emph{root} и не \emph{designated} интерфейсы переходят в статус \textbf{blocked} - BPDU отправляется, но пользовательский траффик не передается.
	
	\subsection{Состояния портов}
	1. \textbf{Disabled} - STP на интерфейсе не работает.
	
	2. \textbf{Listening} - после включения, в этом состоянии находится временно, получает и отправляет \texttt{BPDU}, для пользовательского трафика заблокирован.
	
	3. \textbf{Learning} - пользовательский траффик может \underline{передаваться, но не приниматься}, заполняется таблица маршрутизации.
	
	4. \textbf{Forwarding} - пользовательский траффик \underline{принимается и передается}.
	
	В промежуточных состояниях 2 и 3 коммутатор может находиться не более 15 сек. по умолчанию.
	
	Если 10 сек не поступают сообщения, признается изменение топологии.
	
	\subsection{Оптимизации для использования в современных сетях}
	\textbf{PVST, MST}
	\paragraph{PVST+} - строит несколько независимых графов.
	\paragraph{RSTP} - оптимизация, ускоряющая работу протокола, меньше полагаемся на таймеры.
\end{document}