\documentclass[a4paper,10pt]{article}
\usepackage{amsmath,amssymb,amsfonts,amsthm}
\usepackage{tikz}
\usepackage [utf8x] {inputenc}
\usepackage [T2A] {fontenc} 
\usepackage[russian]{babel}
\usepackage{cmap} 
\usepackage{ gensymb }
% Так ссылки в PDF будут активны
\usepackage[unicode]{hyperref}
\usepackage{ textcomp }
\usepackage{indentfirst}
\usepackage[version=3]{mhchem}

% вы сможете вставлять картинки командой \includegraphics[width=0.7\textwidth]{ИМЯ ФАЙЛА}
% получается подключать, как минимум, файлы .pdf, .jpg, .png.
\usepackage{graphicx}
% Если вы хотите явно указать поля:
\usepackage[margin=1in]{geometry}
% Или если вы хотите задать поля менее явно (чем больше DIV, тем больше места под текст):
\usepackage[DIV=20]{typearea}

\usepackage{fancyhdr}

\newcommand{\bbR}{\mathbb R}%теперь вместо длинной команды \mathbb R (множество вещественных чисел) можно писать короткую запись \bbR. Вместо \bbR вы можете вписать любую строчку букв, которая начинается с '\'.
\newcommand{\eps}{\varepsilon}
\newcommand{\bbN}{\mathbb N}
\newcommand{\dif}{\mathrm{d}}

\newtheorem{Def}{Определение}
\title{Климанов. Лекция 5}

\begin{document}
	\maketitle
	\section{Стандарты Ethernet}
	\emph{NBase} и \emph{NBase-T} - стандарты Ethernet с промежуточными скоростями.
	Возникла проблема, что скорость в \emph{1GBit} мало, а \emph{10Gbit} - избыточно (стоимость электроники).
	\textit{У всех коммутаторов есть ограничения на размер CAM-таблицы ($ \approx 8K $ для самых простых, $ 64k,\ 128K $ для более продвинутых)}. При переполнении таблицы начинают удаляться старые записи или перестают добавляться новые, что все чаще приводит к ситуации, когда каждый фрейм обрабатывается как unknown unicast (рассылаться по всем портам), что ведёт к падению производительности и проблемам с безопасностью.
	\section{Прозрачность оптоволокна}
	\begin{figure}[h]
		\includegraphics[width=0.8\linewidth]{l5_1}
		\caption{$Кадр$}
	\end{figure}
	У Оптоволокна есть несколько \emph{окон прозрачности}, ообычно:
	\begin{enumerate}
		\item 850 нм
		\item 1310 нм
		\item 1350 нм
	\end{enumerate}
	Вообще, существует достаточно многов видов оптоволокна с совершенно разными окнами прозрачности, в том числе прозрачное и на очень широких диапазонах длин волн, но такие виды оптоволокна стоят сильно дороже стандартного с тремя окнами прозрачности.
	\newpage
	Сигналы разных длин волн загоняются в оптоволокно при помощи \emph{мультиплексоров}. 
	\begin{enumerate}
		\item \emph{WDM} - стандартный, обычная призма, которая собирает и разбирает свет разных длин в один.
		\item \emph{CWDM} - для широкого диапазона адресов.
		\item \emph{DWDM} - для узкого диапазона адресов.(?)
		\item \emph{HDWDM} - повышенной плотности - соверменный	
	\end{enumerate}

	\section{Протокол IP}
	Число, 4 байта \texttt{192.168.1.1}
	
	Преимущества над MAC: можно выполнять \emph{агрегацию} IP-адресов (адреса одной подсети принадлежат одному диапазону), что позволяет сильно упростить жизнь маршрутизрующего оборудования.
	
	Сразу после появления IP-адреса делились на 4 диапазона по первому байту:
	\begin{enumerate}
		\item \textbf{A} 1-126 - Unicast (хвостовая часть последние 3 байта, сетевая - первый байт)\
		\item 127 - особый случай, для локальной сети
		\item \textbf{B} 128-191 - Unicast (сетевая - первые 2 байта, хвостовая - два послежних)
		\item \textbf{C} 192-223 - Unicast (сетевая - первые 3 байта, хвостовая - последний)
		\item \textbf{D} 224-239 - Multicast
		\item \textbf{E} - экспериментальный
		\item \texttt{255.255.255.255} - Broadcast
	\end{enumerate}
	\emph{Адреса принадлежат одной подсети, если их сетевая часть совпадает}. Классовость раньше позволяла узнать, какие байты идентифицируют сеть.
	
	Сейчас система IP-адресов \underline{бесклассовая}, для определения байт сети используется \emph{маска}. 
	
	Subnet mask 32 bit - записывается в особом виде \texttt{11111....1110000000...000}, в двоичном виде первыми стоят единицы, затем нули. Таким образом, для определения сетевой части IP-адреса используется логическое \textbf{И} между маской и IP-адресом.
	
	Адреса диапазонов \texttt{10.0.0.0/8}, \texttt{172.16.0.0/16-172.31.0.0/16}, \texttt{192.168.0.0/24-192.168.0.0/16} - зарезервированы для использования в локальных сетях и не ведут в интернет.
	
	\texttt{127.0.0.0/8} - замыкание \emph{loopback}, в интернет тоже не ведут.
\end{document}