\documentclass[a4paper,10pt]{article}
\usepackage{amsmath,amssymb,amsfonts,amsthm}
\usepackage{tikz}
\usepackage [utf8x] {inputenc}
\usepackage [T2A] {fontenc} 
\usepackage[russian]{babel}
\usepackage{cmap} 
\usepackage{ gensymb }
% Так ссылки в PDF будут активны
\usepackage[unicode]{hyperref}
\usepackage{ textcomp }
\usepackage{indentfirst}
\usepackage[version=3]{mhchem}

% вы сможете вставлять картинки командой \includegraphics[width=0.7\textwidth]{ИМЯ ФАЙЛА}
% получается подключать, как минимум, файлы .pdf, .jpg, .png.
\usepackage{graphicx}
% Если вы хотите явно указать поля:
\usepackage[margin=1in]{geometry}
% Или если вы хотите задать поля менее явно (чем больше DIV, тем больше места под текст):
\usepackage[DIV=20]{typearea}

\usepackage{fancyhdr}

\newcommand{\bbR}{\mathbb R}%теперь вместо длинной команды \mathbb R (множество вещественных чисел) можно писать короткую запись \bbR. Вместо \bbR вы можете вписать любую строчку букв, которая начинается с '\'.
\newcommand{\eps}{\varepsilon}
\newcommand{\bbN}{\mathbb N}
\newcommand{\dif}{\mathrm{d}}

\newtheorem{Def}{Определение}
\title{Климанов. Лекция 4}

\begin{document}
	\maketitle
	
	\section{Коммутация. Процесс коммутации}
	
	\subsection{MAC - адреса}
	Основной тип - половина байт - id производителя, вторая - id устройства.
	
	Записываются 16-ричными числами, существует несколько варианта записи:
	
	1. Через \texttt{:}. E.g. \texttt{00:11:22:33:44:aa}
	
	2. Через \texttt{-}. E.g. \texttt{00-11-22-33-44-44}
	
	3. (У Cisco-systems) По 3 байта: \texttt{0011.2233.44aa}
	
	Также существуют \emph{unicast}-адреса:
	\subsubsection{broadcast-адреса}
	
		4. Широковещательные адреса, не могут быть использованы как адреса отправителя и  не назначаются в качестве адреса устройства. Используются как адреса некоторого сегмента сети: \texttt{ff.ff.ff.ff.ff.ff}. 
		Просто ставится в качестве адреса назначения. 
		
		\textbf{Multicast} - сообщение получают только те, кто подписался на "рассылку". Общий вид \texttt{*F.**.**.**.**.**}
		
		\textbf{Broadcast} - сообщение получается ВСЕМИ (карта не отбрасывает такой адрес). Общий вид: \texttt{ff.ff.ff.ff.ff.ff}
	
	\subsection{Мост}
	\textit{Широковещательный домен - часть сети, в которой broadcast-фрейм будет получен всеми узлами сети.}
	
	\emph{Мост (Bridge)} - устройство, работающее на 2 уровне модели OSI. Мост позволяет соединить 2 сегмента сети.
	
	Мост строит \emph{мостовую таблицы (MAP/CAM)}, таким образом обучаясь.
	
	
	\begin{tabular}[h]{|c|c|c|}
		\hline 
		PORT & MAC & time \\ 
		\hline 
		&  &  \\ 
		\hline 
	\end{tabular} 
	
	В случае получения фрейма, адрес которого отсутствует в мостовой таблице (\emph{unknown unicast}) или \emph{broadcast}-фрейм, коммутатор отправляет его по всем портам кроме порта получения (т.к. там уже все получили).
	
	Отправитель отправляет фрейм. Мост смотрит на MAC-адрес, заполняет порт, MAC-адрес отправителя и время получения. При отправлении ответа, мост анализирует свою таблицу, ищет в таблице совпадающий MAC-адрес и переправляет ответ в соответствующий порт в соответствующий сегмент сети. Если нужного адреса нет, фрейм отбрасывается. Таким образом, мот уменььшает коллизионный домент, тк отделяет локальный трафик и снижает нагрузку на сеть.
	
	Значения времени в таблице нужны для проверки актуальности записей таблицы. Это позволяет обновлять ее в случае изменений в сети и удалять устаревшие записи (о ставшими неактивными узлах). Ориентировочное характерное время удаления $ t\approx1\ min $
	
	Мост не изменяет \underline{широковещательный домен}.
	
	Мост делит \underline{коллизионный домен} на 2 части (выступает в качестве обработчика коллизий как обычный узел). Не сообщает узлу-отправителю при отбрасывании фрейм (Ethernet же не гарантирует доставки, ага).
	
	Switch/коммутатор \underline{является мостом}.
	
	\newpage
	
	\subsection{Современное состояние сети}
	Было замечено, что уменьшение размера домена увеличивает пропускную способность сети, потому современные сети находятся в состоянии \emph{микросегментации} - устройства по-одному подключены к портам коммутатора, таким образом:
	
	\begin{enumerate}
		\item Коллизионный домен сократился до 2 устройств.
		\item Стал возможен режим полного дуплекса (повторители и концентраторы его не поддерживали).
	\end{enumerate}
	А значит, в современных сетях, построенных с использованием коммутатора \underline{коллизии невозможны в принципе}.
	
	За счет использования коммутаторов все современные сети имеют \underline{топологию звезды} или \underline{сложной звезды}.
	
	За счет всего это основной принцип \emph{CSMA/CD} постепенно отходит в прошлое и становится ненужным, остаётся только "работа из коробки". \textit{Ethernet довольно сильно изменился со временем, появилось много новых опций, в том числе гарантирующих доставку.}
	
	\subsection{Режимы работы коммутатора}
	\subsubsection{store and forward}
	Коммутатор получает фрейм, записывает его в буффер, только после получения фрейма и проверки контрольной суммы, начинается его поиск в таблице коммутации.
	
	Был популярен раньше, когда вероятность возникновения ошибки была большой.
	
	\subsubsection{cut through}
	Получаем первые 6 байт (MAC получателя), начинаем поиск в мостовой таблице, сразу начинаем передачу по адресу.
	
	Преимущество: минимальная задержка. Эффективен в сетях с низкой вероятностью ошибок передачи и где нет коллизий.
	
	\subsubsection{fragment free}
	Получаем 64 байта - первые "ненадёжные" байты фрейма, при передачи которых возможна коллизия, остальные байты при правильном проектировании сети считаются "безопасными". Проверяем на ошибки передачи, если их нет, то начинаем передавать дальше.
	
	\textit{	Коммутаторы в отличие от концентраторов позволяют подключать в один сегмент сети разноскоростные устройства (за счет наличия у коммутатора буффера). }
	
	
\end{document}