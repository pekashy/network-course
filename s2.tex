\documentclass[a4paper,12pt]{article}
\usepackage{amsmath,amssymb,amsfonts,amsthm}
\usepackage{tikz}
\usepackage [utf8x] {inputenc}
\usepackage [T2A] {fontenc} 
\usepackage[russian]{babel}
\usepackage{cmap} 
\usepackage{ gensymb }
% Так ссылки в PDF будут активны
\usepackage[unicode]{hyperref}
\usepackage{ textcomp }
\usepackage{indentfirst}
\usepackage[version=3]{mhchem}

% вы сможете вставлять картинки командой \includegraphics[width=0.7\textwidth]{ИМЯ ФАЙЛА}
% получается подключать, как минимум, файлы .pdf, .jpg, .png.
\usepackage{graphicx}
% Если вы хотите явно указать поля:
\usepackage[margin=1in]{geometry}
% Или если вы хотите задать поля менее явно (чем больше DIV, тем больше места под текст):
% \usepackage[DIV=10]{typearea}

\usepackage{fancyhdr}

\newcommand{\bbR}{\mathbb R}%теперь вместо длинной команды \mathbb R (множество вещественных чисел) можно писать короткую запись \bbR. Вместо \bbR вы можете вписать любую строчку букв, которая начинается с '\'.
\newcommand{\eps}{\varepsilon}
\newcommand{\bbN}{\mathbb N}
\newcommand{\dif}{\mathrm{d}}

\newtheorem{Def}{Определение}
\title{Климанов. Семинар 2}
\begin{document}
	\maketitle
	... (разговоры про обжимания кабелей)
	\section{Команды}
	 \begin{enumerate}
	 	\item \texttt{enable} - включает режим повышенных приghbdtktubq
	 	\item \texttt{disable} - отключает
	 	\item \texttt{show} - показывает информацию о системе
	 	\subitem (...)
	 	\subitem Сколько работает система
	 	\subitem Количество оперативной памяти, информация о ней, разрпядность
	 	\subitem Частота процессора
	 	\subitem Список интерфейсов
	 	\subitem \emph{Конфигурационный регистр} - отвечает за то, каким образом загружается устройство.
	 	
	 	\texttt{show run} - показывает \emph{startup-config}. При запуске устройства конфигурация из \emph{startup-config} записывется в \emph{running-config}. При перезагрузке устройства информсмация в \emph{runnin config} будет утеряна, поэтому для сохранения сделанных в ней изменений стоит записать ее в \emph{startup-config}:
	 	
	 	\texttt{copy running-config startup-config}
	 	
	 	Или (не везде работает) есть сокращенная команда:
	 	
	 	\texttt{write}
	 	
	 	Для того что бы просто сохранить файл конфигурации на диск:
	 	
	 	\texttt{copy running-config flash:filename}
	 	
	 	Список уже сохраненных конфигов:
	 	
	 	\texttt{copy running-config flash:?}
	 	
	 	Вывести список файлов в памяти:
	 	
	 	\texttt{show flash}
	 	
	 	Для того что бы понять загрузку CPU (список процессов, загрузка процессора определенным устройством за промежутки времени):
	 	
	 	\texttt{show proc cpu}
	 	
	 	График загрузки процессоров:
	 	
	 	\texttt{show proc cpu history}
	 		
	 		Считается нормальной загрузка \emph{средняя} процессора меньше \emph{60} процентов.
	 \end{enumerate}
\end{document}