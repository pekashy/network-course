\documentclass[a4paper,10pt]{article}
\usepackage{amsmath,amssymb,amsfonts,amsthm}
\usepackage{tikz}
\usepackage [utf8x] {inputenc}
\usepackage [T2A] {fontenc} 
\usepackage[russian]{babel}
\usepackage{cmap} 
\usepackage{ gensymb }
% Так ссылки в PDF будут активны
\usepackage[unicode]{hyperref}
\usepackage{ textcomp }
\usepackage{indentfirst}
\usepackage[version=3]{mhchem}

% вы сможете вставлять картинки командой \includegraphics[width=0.7\textwidth]{ИМЯ ФАЙЛА}
% получается подключать, как минимум, файлы .pdf, .jpg, .png.
\usepackage{graphicx}
% Если вы хотите явно указать поля:
\usepackage[margin=1in]{geometry}
% Или если вы хотите задать поля менее явно (чем больше DIV, тем больше места под текст):
\usepackage[DIV=20]{typearea}

\usepackage{fancyhdr}

\newcommand{\bbR}{\mathbb R}%теперь вместо длинной команды \mathbb R (множество вещественных чисел) можно писать короткую запись \bbR. Вместо \bbR вы можете вписать любую строчку букв, которая начинается с '\'.
\newcommand{\eps}{\varepsilon}
\newcommand{\bbN}{\mathbb N}
\newcommand{\dif}{\mathrm{d}}

\newtheorem{Def}{Определение}
\title{Климанов. Лекция 2}

\begin{document}
	\maketitle
	Локальные сети.
	
	Сегодня - в основном про первый уровень модели \emph{OSI}
	
	В современных проводных локальных сетях основная технология - \emph{Ethernet}.
	
	\section{На физическом уровне:}
	\begin{enumerate}
		\item Витая пара \emph{UTP} - \emph{неэкранированная витая пара}
		\item Оптика
		\item Твинаксиальный кабель - \emph{twinax} (в основном в Data-центрах)
	\end{enumerate}
	На заре появления всего самым популярным был \emph{коаксиальный кабель}.
	
	\subsection{Коаксиальный кабель}
	Плюсы коаксиального кабеля:
	\begin{enumerate}
		\item Недорогой
		\item Все устройства подключаются к нему напрямую.
		
		Такой способ подключения называется \emph{шина (bus)}
		
		\emph{BNC} - специальный разъем для такого способа подключения
	\end{enumerate}

	Минусы коаксиального кабеля:
	\begin{enumerate}
		\item Если в любом месте сети появлялся разпыв, переставала работать \underline{вся сеть}.
		
		На границах сегментов кабеля стояли специальные устройства - \emph{терминаторы}, которые гасили остаточный сигнал (?).
		
		Ограничения размеров сегмента для двух популярных диаметров (тонкого и толстого) - \textit{200 и 500м}
		
		\item Коаксиальный кабель не позволяет передавать данные 2 устройствам одновременно
		\item Теоретический предел скорости - \textbf{10 Mbps}
	\end{enumerate}
	
	\emph{Repeater} - устройство, повторяющее сигнал, посзволяющее обойти ограничение на длину сегмента кабеля.
	
	\emph{5-4-3} - правила построения сетей на коаксиальном кабеле
	
	\subsection{Витая пара}
	Виды:
	\begin{enumerate}
		\item \emph{UTP} - неэкранированная витая пара
		\item \emph{STP} - экранированная витая пара
		\item \emph{FTP} - фольгированная витая пара
	\end{enumerate}

	\subsubsection{UTP}
	\emph{Unshielded Twisted Pair}
	
	В общаге мы подключаемся через нее)
	
	4 пары, 8 проводников
	
	\emph{Twisted} - 4 пары перекручены между собой на всей длине, чтобы уменьшить влияние передающихся сигналов друг на друга и приём сторонних сигналов.
	


	Маркировка позволяет правильно прикрепить коннектор \emph{8P8C} (В народе неправильное название - RJ-45(на самом деле наш разъем отличается, так как стандарт RJ-45 задает другое правило расположения проводов))
	
	Пары скручиваются с переменных, разный у разных пар шагом.
	
	\subsection{STP}
	Внутри пары каждая жила экраникуется.
	
	\subsection{STP}
	Еще и фольгируется
	
	
	\emph{AWG} - сколько витую пару тянули, чем больше, тем тоньше пара.
	
		Маркировка портов в RJ-45 разъеме:
	\begin{enumerate}
		\item Бело-зелёный
		\item Зеленый
		\item Бело-оранжевый
		\item Синий
		\item Бело-синий
		\item Оранжевый
		\item Бело-коричневый
		\item Коричневый
	\end{enumerate}
	
	\subsection{Разводка пары}
	\subsubsection{Стандарт ICO Base-TX}
	В разъеме контакты \emph{1, 2, 3} - \emph{Tx}
	
	\emph{3, 6} - \emph{Rx}
	
	В \emph{Tx} и \emph{Rx} необходимо подключать \underline{разные} пары, иначе тк, одинаковые между собой не скручены, мы получим ОГРОМЕННУЮ антенну.
	
	Контакты \emph{4 и 5} использовались для подключения \emph{RJ-21} - разъема "как у домашнего телефона", которые раньше использовались как разъемы для подключения переговорных устройств в сетях.
	
	\textit{Современный стандарт \emph{СКС} уже не требует их. Например, в гигабитных сетях они уже используются для передачи данных.}
	
	При подключении витой пары к какому-либо сетевому оборудованию, мы используем \emph{прямой патчкорд}.
	На коммутаторах что бы не использовали прямые кабели, Rx и Tx поменяли местами.
	
	Кабель-кроссовер (нужен для соединения двух свитчей между собой):
		\begin{enumerate}
		\item Бело-оранжевый
		\item Оранжевый
		\item Бело-зеленый
		\item Синий
		\item Бело-синий
		\item Зеленый
		\item Бело-коричневый
		\item Коричневый
	\end{enumerate}

	В современных компьютерах существуют раскладки \emph{MDI} и \emph{MDIX}, которая позволяет определитьть, что находится с другой стороный и \emph{переворачивает разъем}, что позволяет \underline{соединять прямыми кабелями \emph{всё}}.
	
	\subsubsection{Категория кабеля}
	Определяет качества кабеля.
	
	Существует 7 категорий:
	\begin{enumerate}
		\item 1-2 не предназначены для работы с ethernet
		\item 5 (обычная неэкранированная витая пара) - на ней построены большая часть сетей
		\item 5e - оптимищация 5
		\item 6 и 6A - экранированная витая пара, ползволяет передавать на более высоких частотах, передавать с большими скоростями
	\end{enumerate}
	
	\textit{А прямой зависимости между максимальной скоростью передачи и максимальной частотой для кабеля - \underline{нет}.}
	
	\subsubsection{Типы волокн}
	Тип волокна в кабеле определяет то, сколько световых мод он пропускает, что определяет дальность передачи сигнала.
	
	\emph{Одномодовое \underline{SM}} и \emph{многомодовое \emph{SM}} волокно различаются диаметром сердечника:
	
	\underline{7-10 мкм} - одномодовое. Пропускает только один пучок света.
	
	\underline{50-62,5 мкм} - многомодовое. Передает несколько мод, входящих в него под разными углами.
	
	Разницы в цене между волокнами нет, разница в цене существует для \emph{трансиверов}. Стандарты разъемов трансиверов:
	
	\emph{SFP} - позволяет передавать до 1Gbps
	
	\emph{SFP+} - 10 Gbps
	
	\emph{QSFP} - 4 Gbps
	
	\emph{QSFP+} - 40 Gbps
	
	\textit{В современных датацентрах можно встретить скорости 40Gbps и 100Gbps и даже 400Gbps (все еще дорого)}
	
	\textit{Сегодня можно встретить переходники с оптических разъемов на медные RJ-45}.
	
	\subsubsection{Физические принципы оптических кабелей}
	 \emph{SFP:} При помощи призмы собираем сигналы нескольких длин волн в кабель, а потом при помощи другой призмы раскладываем обратно.
	 
	 \emph{SFP28:} Позволяет передовать данные на скорости 25Gbps сигнал на одной длине волны. Сигнал с одной длиной волны проще мультиплексировать.
\end{document}