\documentclass[a4paper,10pt]{article}
\usepackage[table,xcdraw]{xcolor}
\usepackage{multirow}
\usepackage{amsmath,amssymb,amsfonts,amsthm}
\usepackage{tikz}
\usepackage [utf8x] {inputenc}
\usepackage [T2A] {fontenc} 
\usepackage[russian]{babel}
\usepackage{cmap} 
\usepackage{ gensymb }
% Так ссылки в PDF будут активны
\usepackage[unicode]{hyperref}
\usepackage{ textcomp }
\usepackage{indentfirst}
\usepackage{adjustbox}
\usepackage[version=3]{mhchem}
\usepackage{MnSymbol,wasysym}

% вы сможете вставлять картинки командой \includegraphics[width=0.7\textwidth]{ИМЯ ФАЙЛА}
% получается подключать, как минимум, файлы .pdf, .jpg, .png.
\usepackage{graphicx}
% Если вы хотите явно указать поля:
\usepackage[margin=1in]{geometry}
% Или если вы хотите задать поля менее явно (чем больше DIV, тем больше места под текст):
\usepackage[DIV=20]{typearea}

\usepackage{fancyhdr}

\newcommand{\bbR}{\mathbb R}%теперь вместо длинной команды \mathbb R (множество вещественных чисел) можно писать короткую запись \bbR. Вместо \bbR вы можете вписать любую строчку букв, которая начинается с '\'.
\newcommand{\eps}{\varepsilon}
\newcommand{\bbN}{\mathbb N}
\newcommand{\dif}{\mathrm{d}}

\newtheorem{Def}{Определение}
\title{Климанов. Лекция 13}

\begin{document}
	\maketitle
	\section{Proxy Arp}
	\begin{figure}[h!]
		\centering
		\includegraphics[width=0.8\linewidth]{13-1}
	\end{figure}
	Проблема: если установим неправильную маску для Server, мы можем потерять контроль над \emph{Server}. Например, если мы поставим маску \texttt{/16} для нашего сервера, то он будет применять ее в том числе и для адреса источника, который после применения маски будет выглядеть как находящийся в той же сети. Соответственно, вместо пересылки ответа через роутер, будет составляться ARP-запрос, на который очевидно не будет получен ответ => \underline{попытка установления TCP провалится}.
	\textbf{Proxy Arp} позволяет роутеру выступать в качестве "источника", он принимает адресованный источнику ARP-запрос, пересылает свой адрес в ответ и, соответственно, обеспечивает доставку информации до источника.
	
	\textit{Не должен использоваться на постоянной основе, создает слишком большую нагрузку на сеть и CPU маршрутизаторов}
	
	\section{DHCP}
	\textbf{Dynamic Host Configuration Protocol} - позволяет автоматически настроить IP адреса и еще некоторые параметры.\\
	
	Основные 4 типа сообщений DHCP, которые используются для получения параметров:
	\begin{table}[h]
		\begin{tabular}{|l|l|l|}
			\hline
			Номер & Клиент                                                                                                                                                   & Сервер                                     \\ \hline
			1     & \begin{tabular}[c]{@{}l@{}}DHCP-Discover - широковещательная рассылка,\\ поиск DHCP-серверов\end{tabular}                                                &                                            \\ \hline
			2     &                                                                                                                                                          & DHCP-offer - отправляет свое "предложение" \\ \hline
			3     & \begin{tabular}[c]{@{}l@{}}Принимается "offer". Отправляется DHCP-request - \\ перепроверка и запрос на использование параметров сервера\end{tabular} &                                            \\ \hline
			4     &                                                                                                                                                          & ACK - подтверждение                        \\ \hline
		\end{tabular}
	\end{table}


	После этой цепочки за клиентом закрепляются выданные параметры на время \emph{lease-time (?)}.

	\textit{Адреса могут даваться на долгое время в стабильных сетях}
		\newpage
	\subsection{Еще сообщения DHCP}
	\textbf{Отказ DHCP}
	
	Если после получения подтверждения (\texttt{DHCPACK}) от сервера клиент обнаруживает, что указанный сервером адрес уже используется в сети, он рассылает широковещательное сообщение отказа DHCP (\texttt{DHCPDECLINE}), после чего процедура получения IP-адреса повторяется. Использование IP-адреса другим клиентом можно обнаружить, выполнив запрос ARP.
	
	Сообщение отмены DHCP (\texttt{DHCPNAK}). При получении такого сообщения соответствующий клиент должен повторить процедуру получения адреса.
	
	\textbf{Освобождение DHCP}
	
	Клиент может явным образом прекратить аренду IP-адреса. Для этого он отправляет сообщение освобождения DHCP (DHCPRELEASE) тому серверу, который предоставил ему адрес в аренду. В отличие от других сообщений DHCP, DHCPRELEASE не рассылается широковещательно.
	
	\textbf{Информация DHCP}
	
	Сообщение информации DHCP (\texttt{DHCPINFORM}) предназначено для определения дополнительных параметров TCP/IP (например, адреса маршрутизатора по умолчанию, DNS-серверов и т. п.) теми клиентами, которым не нужен динамический IP-адрес (то есть адрес которых настроен вручную). Серверы отвечают на такой запрос сообщением подтверждения (\texttt{DHCPACK}) без выделения IP-адреса. 
	
	\subsection{DHCP сервер в сети}
	
	\begin{figure}[h!]
		\centering
		\includegraphics[width=0.5\linewidth]{13-2}
		\caption{Пример топологии}
	\end{figure}
	
	DHCP запросы проходят через роутер на DHCP-сервер, подключенный к нему (широковещательные пакеты (определенного вида) переправляются роутером к DHCP). Для управления адресным пространством может быть, например,  \texttt{IPAM}
	
	\subsection{DHCP-Snooping}
	Прослушиваем процесс обмена DHCP сообщений, составляем свою собственную базу, включающую IP и MAC, отбрасываем все стремные пакеты.
	
	Защищает от атаки вида "Подмена IP-адресов".
	
	\section{Топологии в современных датацентрах}
	(...)
	Core - внутренний уровень
	Distribution 
	Acc/QoS - внешний
	
	\subsection{Clos}
	Делим коммутаторы на 2 категории: \texttt{Leaf} и \texttt{Spine}, Обычно берут немного \texttt{Spine} и много \texttt{Leaf} и соединяют каждый \texttt{Leaf} к каждому \texttt{Spin'у}. Все подключения к сети и сети к другим сетям происходит на уровне \texttt{Leaf}, эта технология позволяет удобно масштабировать сеть.
	
	\subsubsection{VXLAN}
	Тоннель между Leaf'ами
	\texttt{Spanning tree} не используется в таких сетях, вместо него технология \textbf{vxlan}
\end{document}