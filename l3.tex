\documentclass[a4paper,10pt]{article}
\usepackage{amsmath,amssymb,amsfonts,amsthm}
\usepackage{tikz}
\usepackage [utf8x] {inputenc}
\usepackage [T2A] {fontenc} 
\usepackage[russian]{babel}
\usepackage{cmap} 
\usepackage{ gensymb }
% Так ссылки в PDF будут активны
\usepackage[unicode]{hyperref}
\usepackage{ textcomp }
\usepackage{indentfirst}
\usepackage[version=3]{mhchem}

% вы сможете вставлять картинки командой \includegraphics[width=0.7\textwidth]{ИМЯ ФАЙЛА}
% получается подключать, как минимум, файлы .pdf, .jpg, .png.
\usepackage{graphicx}
% Если вы хотите явно указать поля:
\usepackage[margin=1in]{geometry}
% Или если вы хотите задать поля менее явно (чем больше DIV, тем больше места под текст):
\usepackage[DIV=20]{typearea}

\usepackage{fancyhdr}

\newcommand{\bbR}{\mathbb R}%теперь вместо длинной команды \mathbb R (множество вещественных чисел) можно писать короткую запись \bbR. Вместо \bbR вы можете вписать любую строчку букв, которая начинается с '\'.
\newcommand{\eps}{\varepsilon}
\newcommand{\bbN}{\mathbb N}
\newcommand{\dif}{\mathrm{d}}

\newtheorem{Def}{Определение}
\title{Климанов. Лекция 3}

\begin{document}
	\maketitle
	\section{Топологии}
	Существует несколько топологий, с помощью которых можно строить сеть.
	1. \textbf{Шина}
	
	2. \textbf{Кольцо (Wheel)}
	Чаще можно встретить в сетях, назыывающихся \emph{MAN} - сети размером с мегаполис. Крупные сети, расположенные на ограниченных географически пространствах, соединяют в себе свойства глобальных и локальных сетей.
	
	3. \textbf{Двойное кольцо}
	Используется когда требуется повышенная отказоустойчивость (как обычное кольцо, только узлы и соединения продублированы). 
	
	Данные могут передаваться в одну сторону по одному кольцу и в другую по другую.
	
	\emph{Ethernet} кольца не поддерживает, так что обычно можно встретить в крупных сетях.
	
	3. \textbf{Full mash} Самая связная топология, лучшее решение в плане надёжности, но очень дорогая в плане поддержки и вообще. 
	
	4. \textbf{Звезда} Чаще всего встречается в современных сетях на базе Ethernet. Состоит из центральных устройств, к которым подключены остальные. Меньшие "зёздочки" могут связываться тоже при помощи "звёздной структуры.
	
	Также используются промежуточные устройства \emph{hub, switch}.
	
	В локальных сетях чаще связывается при помощи \emph{витой пары}, в современных дата-центрах при помощи \emph{оптики} или \emph{коаксиального кабеля Twinax}.
	
	Изначально на портах серверных устройств использовались трансиверы \emph{sfp  (qsfp, sfp+ ...)}, что позволяло передавать сигнал по различным видам волокн, но это было дороговато.
	Сейчас для дешевизны используется решение - twinax кабели фиксированной длины с уже замонтироваными sfp на концах. 
	
	При помощи \emph{brakeout - кабеля} можно перевести в 4 независимых порта на \emph{10 Gbps} один порт на \emph{40 Gbps}
	
	Устройства компании \emph{Azista} позволяют сделать наоборот.
	
	\section{Ethernet}
	Причина успеха - дешевизна и простота.
	
	Данные передаются с помощью блоков - \emph{кадров}:
	
	\subsection{Кадр}
	\begin{figure}[h]
		\includegraphics[width=0.95\linewidth]{3_1}
		\caption{$Кадр$}
	\end{figure}
	\newpage
	\texttt{Преамбула | Destination adress 5 bytes (DAM) | адрес получателя 5 bytes (SAM) | EtherType T/L длина данных и тип протокола 2bytes | Data 46-1500 bytes /FCS 4 bytes}
	Суммарно - 64-1518 байт
	
	\begin{enumerate}
		\item Преамбула - не является частью Eth-заголовка, определяет начало Ethernet-фрейма
		\item MAC - адрес назначения (на лекции сказали что 5 байт)
		\item Адрес отправителя (на лекции сказали что 5 байт)
		\item EtherType - идентификатор протокола
		\item Payload - данные
		\item FCS (Frame Check Sequences) – 4 байтное значение CRC используемое для выявления ошибок передачи. Вычисляется отправляющей стороной, и помещается в поле FCS. Принимающая сторона вычисляет данное значение самостоятельно и сравнивает с полученным
	\end{enumerate}
	
	\subsection{Maс - адрес}
	
	Состоит из 2 частей по 3 байта, каждая из которых актуальна, где первая - идентификатор компании, а вторая - идентификатор устройства.
	
	\subsection{Ethernet}
	CarriedSenseMultipleAccess/CollisonDetection - основной принцип Ethernet
	
	1. \textbf{Multiple Access} Среда с множественным доступом, к одному источнику возможно подключение нескольких устройств
	
	2. \textbf{Caried Sense} - сетевой картой выполняется прослушивание, свободна ли среда. Если среда занята, устройство ждет, пока среда освободится.
	
	3. \textbf{Collision Detection} 
	
	\emph{Коллизия} - когда два узла начинаютт передачу одновременно.
	
	В процессе передачи среда продолжает прослушиваться. В случае если все ОК, то фрейм передается и удаляется из ОЗУ. Если коллизия произошла, то:
	\begin{enumerate}
		\item Передача приостанавливается.
		\item Запускается таймер на время: 
		\[ L\cdot 512 \]
		Где $ L=random[1;2^n-1] $, где $ n $ - номер попытки передачи, $ 512 $ - необходимое для передачи 512 байт время.
		Если не удалось передать за 16 попыток, генерируется исключение. $ n $ растет до 4-5, \underline{не до 16}
	\end{enumerate}
	Конкурентный доступ такого вида позволил максимально упростить оборудование.
	
	\subsection{Сети построенные на Coax}
	
	У Coax-кабеля существуют ограничения на длину сегмента (200 для тонкого и 500 для толстого). Для увеличения длины сегмента используются \emph{повторители}, которые очищают и усиливают сигнал. При этом на \emph{колизионный домен} - участок сети, подверженный одной коллизии они \underline{никак не влияют}. С \emph{хабами} - многопортовыми - та же фигня.
	
	\subsection{Логика работы Ethernet-станций}
	Допустим, А знает mac-адрес B. A формирует сегмент, отправляет его в сеть, этот фрейм получают \underline{все узлы, подключенные к локальному сегменту}. \underline{В теории}, сетевые карты сравнивают mac-адрес получателя со своим и если он не совпадает - отбрасывают.
	
	У сетевой карты есть \emph{неразборчивый} режим, который принимает \underline{все пакеты}.
	
	Сетевая карта \underline{не может} проверить \emph{подлинность отправителя}.
	
	Сеть на хабах: \underline{не защищает от коллизий}, \underline{не решает проблем с безопасностью}, только увеличивает расстояние распостранения сети.
	
	\subsection{Дуплекс}
	Режим передачи данных, бывает два вида:
	
	\begin{enumerate}
		\item \emph{Полудуплекс или хаб-дуплекс} - возможна передача только в одном направлении.
		
		Сеть на coax-кабеле или хабах - всегда полудуплекс.
		
		\item \emph{Полный дуплекс Full Duplex} - возможно передача в двух направлениях, \underline{не бывает коллизий}.
		
		\textit{Шины не поддерживают полный дуплекс}
		
	\end{enumerate}
	
	\subsection{5-4-3}
	\emph{FastEthernet} - 100Mbps Ethernet
	Логично что при использовании хабов увеличивается коллизинный домен, сигналу нужно больше времени, чтобы дойти до более дальней станции, соответственно падает пропускная способность. Поэтому для того что бы с этим бороться и было введено правило \emph{5-4-3} - ограничение на максимальную длину сети на хабах.
	
	Для того что бы быть способным определить коллизию, мы должны передавать фрейм за время не большее, чем он дойдёт до наиболее удаолённого узла сети.ы
	
\end{document}